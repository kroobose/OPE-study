\begin{figure*}[!t]
    \centering
    \scalebox{0.8}{
    \begin{tikzpicture}
        \begin{axis}[
            ybar,
            bar width=.8cm,
            width=5.5cm,
            height=5cm,
            symbolic x coords={行動1, 行動2, 行動3},
            xtick=data,
            axis x line=bottom,
            ymin=0,
            axis y line=left,
            nodes near coords={
                \pgfmathprintnumber[fixed, precision=5]{\pgfplotspointmeta}
            },
            ytick=\empty,
            enlarge x limits=0.3,
            ylabel={行動選択確率},
            title={データ収集方策$\pi_0$},
            title style={yshift=.2cm},
        ]
        \addplot+[
            ybar,
            bar shift=0pt,
            draw=black,
        ]  coordinates {(行動1,0.80) (行動2,0.0) (行動3,0.20)};
        \end{axis}

        \begin{axis}[
            ybar,
            bar width=.8cm,
            width=5.5cm,
            height=5cm,
            symbolic x coords={行動1, 行動2, 行動3},
            xtick=data,
            axis x line=bottom,
            ymin=0,
            axis y line=left,
            nodes near coords={
                \pgfmathprintnumber[fixed, precision=5]{\pgfplotspointmeta}
            },
            ytick=\empty,
            yticklabel style={/pgf/number format/.cd,
                fixed,
                precision=3,
                zerofill,
                scaled ticks=false
            },
            enlarge x limits=0.3,
            ylabel={行動選択確率},
            title={評価方策$\pi$},
            title style={yshift=.2cm},
            at={(6cm,0)},
        ]
        \addplot+[
            ybar,
            bar shift=0pt,
            draw=black,
        ]  coordinates {(行動1,0.0) (行動2,0.40) (行動3,0.60)};
        \end{axis}

        \begin{axis}[
            ybar,
            bar width=.8cm,
            width=5.5cm,
            height=5cm,
            symbolic x coords={行動1, 行動2, 行動3},
            xtick=data,
            axis x line=bottom,
            ymin=0,
            axis y line=left,
            nodes near coords={
                \ifnum\coordindex=1
                    定義不能
                \else
                    \pgfmathprintnumber[fixed, precision=5]{\pgfplotspointmeta}
                \fi
            },
            ytick=\empty,
            yticklabel style={/pgf/number format/.cd,
                fixed,
                precision=3,
                zerofill,
                scaled ticks=false
            },
            enlarge x limits=0.3,
            ylabel={重みの値},
            title={重要度重み$\pi/\pi_0$},
            title style={yshift=.2cm},
            at={(12cm,0)},
        ]
        \addplot+[
            ybar,
            bar shift=0pt,
            draw=black,
        ] coordinates {(行動1,0.0) (行動2,0.0) (行動3,3.0)};
        \end{axis}
    \end{tikzpicture}
    }
    \caption{Assumption~\ref{ass:common-support-ips}(共通サポート)が満たされない例.}
    \label{fig:ips-hist2}
\end{figure*}
